% \input utf8-t1
\documentclass[12pt,titlepage,a4paper]{article}

\usepackage[czech]{babel}
\usepackage[utf8]{inputenc}
\usepackage{hyperref}

\begin{document}

\title{Pybots - herní server pro foo bar}
\author{Josef Kolář}
\date{leden 2016}
\maketitle

\section{Použité technologie}

\subsection{Python}

Python je moderní interpretovaný programovací jazyk, který byl navržen v roce 1991 nizozemským programátorem Guido van Rossumem. Nabízí rozličná programovací paradigma: imperativní, procudurální, funkcionální nebo objektově orientované, které jsem ve své práci použil nejčastěji.

Python je vyvíjen jako open source projekt, jeho zdrojové kódy jsou tedy veřejné a je možné do nich přispět. Jeho výchozí implementace se nazývá ``CPython'' dle jazyka C, ve kterém je implementována. Mezi další jeho alternativní implementace patří ``Jython'' naprogramovaný v jazyce Java nebo ``IronPython'' v prostředích .NET a Mono.

Jednou z velkých výhod Pythonu je jeho rozšířitelnost. Oficiální 
portál pro rozšiřující balíčky 
\href{https:\/\/pypi.python.org\/pypi}{PyPi} aktuálně nabízí okolo 74 tisíc knihoven. Kterýkoliv z těchto balíčků je možno pomocí nástroje pip nainstalovat a používat. Jednou z dalších možností, jak rozšířit jeho funkčnost, je naprogramovat si vlastní rozšíření v jazyce C.

Python je aktuálně vyvíjen ve dvou hlavních větvích; větvi Pythonu verze 2 a verze 3. Motivace pro vydání verze 3 byla především ve sjednocení práce s řetězci (Python verze 2 rozlišoval řetězce ASCII znaků a řetězce Unicode znaků), celočíselného dělení a některých vylepšení syntaxe jazyka.

\subsection{Flask}

Flask je webový framework implementovaný v jazyce Python. Mezi jeho 
přednosti patří vestavěný vývojářský server, plná podpora pro 
unit testování a kvalitní dokumentace. Vzhledem k jeho variabilitě 
a v základu tenké architektuře bylo nutné přidat abstraktní 
vrstvy pro samotné zpracování herních akcích a podobně.

\subsection{HTTP}

HTTP (HyperText Transfer Protocol) je standardizovaný internetový 
protokol pro výměnu HTML kódu. V dnešní době je používán nejen 
k výměně HTML kódu, ale jeho rozšíření MIME \footnote{Multipurpose Internet Mail Extensions}

\end{document}
